\documentclass[journal]{IEEEtran}

\usepackage{float}
\usepackage{amsmath}
\usepackage{subfigure,graphicx}
\usepackage{multicol}
\usepackage{stfloats}
\usepackage[portuguese]{babel}
%\usepackage{hyperref}
\usepackage{booktabs}
\usepackage[protrusion=true,expansion=true]{microtype}
\usepackage{amsfonts,latexsym,amssymb,euscript,xr,amsthm}

\begin{document}

\title{TITULO}

\author{Hugo Veríssimo. hugoverissimo@ua.pt, 124348.}

\markboth{Mineração de Dados em Larga Escala. 5 de Março de 2025. MUDAR MUDAR MUDAR MUDAR MUDAR MUDAR MUDAR}{} %MUDAR % MUDAR

\maketitle

\begin{abstract}
abstrato
\end{abstract}

\IEEEpeerreviewmaketitle

\section{Introdução}

O presente trabalho tem como objetivo modelar e prever o consumo mensal de eletricidade em Espanha, utilizando dados oficiais do Eurostat \cite{eurostat_nrg_cb_em}. A análise de séries temporais neste contexto reveste-se de particular importância, dado o papel estratégico da energia elétrica na economia e na definição de políticas públicas. A capacidade de antecipar padrões de consumo permite uma melhor gestão de recursos, planeamento da produção e tomada de decisões informadas por parte dos agentes do setor energético.

baseia-se em ... ("metodologia e coisas usadas")

CHATGPT:

A metodologia seguida baseia-se em técnicas clássicas de modelação de séries temporais. Inicia-se com uma análise exploratória da série, onde se investigam tendências, sazonalidade e possíveis irregularidades. A estacionariedade é avaliada tanto de forma gráfica como através de testes formais (ADF e KPSS).

Com a série estacionária, procede-se à estimação de vários modelos SARIMA, que incorporam componentes sazonais e não sazonais. A seleção do modelo baseia-se em critérios de informação (AICc), ajustamento visual e diagnóstico dos resíduos. O objetivo final é identificar um modelo que capture adequadamente a estrutura temporal dos dados e permita previsões robustas e fiáveis.



\section{Data and exploratory analysis/data transformation}

a

bla bla falar tbm do haver um outlier em 2020 devido ao covid mas que acabou por n after a performance dos modelos pelo que nao feito necessario qq cuidado especial

\subsection{c}

a

\begin{table}[H]
\centering
\caption{OLD OLD OLD Comparação da Acurácia Entre os Modelos Desenvolvidos e os Modelos do Artigo}
\label{accuracies}
\begin{tabular}{l|cc|cc|c}
\toprule
 & \multicolumn{2}{c|}{Train Set} & \multicolumn{2}{c|}{Test Set} & \\
\textbf{Classificador} & \textbf{Full} & \textbf{Top 15} & \textbf{Full} & \textbf{Top 15} & \textbf{Artigo} \\
\midrule
Random Forest & 99.3 & 98.7 & 99.7 & 98.9 & 98.7 \\
\bottomrule
\end{tabular}
\end{table}

\section{ Model proposals}

 (SARIMA type and/or ETS model and/or GARCH model):

Box  Jenkins methodology if SARIMA models; this includes model identification,
parameter estimation, diagnostic evaluation (parameters correlation,
parameters significance, residual analysis).

\subsection{SARIMA}

b

\section{Future observations forecast}

Includes the analysis of its accuracy, confidence intervals (if possible), use of
bootstrap methodology (if necessary). Accuracy of the forecasts should be
compared.

\section{Results, discussion/conclusions:}

Discussion should include the model choice

\section{Conclusão}

conc

\bibliographystyle{IEEEtran}
\bibliography{refs}

\appendices
\section{Título do Apêndice A}
Auxiliary/not so important results are included in the Appendix.

\section{Título do Apêndice B}
Auxiliary/not so important results are included in the Appendix.

\appendix
\section{OU se for só 1: Título do Apêndice}
Auxiliary/not so important results are included in the Appendix.

\end{document}